%%% template.tex
%%%
%%% This LaTeX source document can be used as the basis for your technical
%%% paper or abstract. Intentionally stripped of annotation, the parameters
%%% and commands should be adjusted for your particular paper - title, 
%%% author, article DOI, etc.
%%% The accompanying ``template.annotated.tex'' provides copious annotation
%%% for the commands and parameters found in the source document. (The code
%%% is identical in ``template.tex'' and ``template.annotated.tex.'')

\documentclass[annual]{acmsiggraph}

\TOGonlineid{45678}
\TOGvolume{0}
\TOGnumber{0}
\TOGarticleDOI{1111111.2222222}
\TOGprojectURL{}
\TOGvideoURL{}
\TOGdataURL{}
\TOGcodeURL{}

\title{Goes Here}

\author{}
\pdfauthor{}

\keywords{}

\begin{document}

%% \teaser{
%%   \includegraphics[height=1.5in]{images/sampleteaser}
%%   \caption{Spring Training 2009, Peoria, AZ.}
%% }

\maketitle

\begin{abstract}


\end{abstract}

\keywordlist

\TOGlinkslist

\copyrightspace

\section{Introduction}
\section{System Description}
\subsection{Input API}
Specification of an optimization problem using our API.
\subsection{Optimization Framework}
Describe four main components and some built-in libraries.
\section{Experiments}
\subsection{Previous Work}
Subsurface scattering. Mechanical properties
\subsection{Textured Model}
Input: a textured 3D model and
	measured Albedo of print material.

Output: (Probably gray-scale) Material arranged for different printers in different formats: STL, fable,Gcode.

Printers: Our printer, Makerbot, Objet, maybe Zcorp.

\subsection{New Examples for Mechanical Properties}
Ball bounce to certain height. Loaded dice.

Printer:objet, ours.

\section{Possible Extension}
UI to specify input deformation.
\section{Conclusion}
\section*{Acknowledgements}

\bibliographystyle{acmsiggraph}
\bibliography{template}
\end{document}
