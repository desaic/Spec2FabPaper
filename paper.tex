%%% template.tex
%%%
%%% This LaTeX source document can be used as the basis for your technical
%%% paper or abstract. Intentionally stripped of annotation, the parameters
%%% and commands should be adjusted for your particular paper - title, 
%%% author, article DOI, etc.
%%% The accompanying ``template.annotated.tex'' provides copious annotation
%%% for the commands and parameters found in the source document. (The code
%%% is identical in ``template.tex'' and ``template.annotated.tex.'')

\documentclass[annual]{acmsiggraph}

\TOGonlineid{45678}
\TOGvolume{0}
\TOGnumber{0}
\TOGarticleDOI{1111111.2222222}
\TOGprojectURL{}
\TOGvideoURL{}
\TOGdataURL{}
\TOGcodeURL{}

\usepackage{caption}
\usepackage{subcaption}

\title{Goes Here}

\author{}
\pdfauthor{}

\keywords{}

\begin{document}

 \teaser{
   \begin{subfigure}[b]{0.3\textwidth}
	\centering
%    	\includegraphics[width=\textwidth]{a}
      	\caption{}
        \label{teaser:original}
        \end{subfigure}
        ~ 
        \begin{subfigure}[b]{0.3\textwidth}
                \centering
%    	\includegraphics[width=\textwidth]{a}
                \caption{}
                \label{teaser:ours}
        \end{subfigure}
        ~
        \begin{subfigure}[b]{0.3\textwidth}
                \centering
%    	\includegraphics[width=\textwidth]{a}
                \caption{}
                \label{teaser:makerbot}
        \end{subfigure}
        ~
        \begin{subfigure}[b]{0.3\textwidth}
                \centering
%    	\includegraphics[width=\textwidth]{a}
                \caption{}
                \label{teaser:objet}
        \end{subfigure}
        \caption{A textured mesh printed with different types of 3D printers.
        	(a)original model.(b)printed with our printer.(c)makerbot.
        	(d)objet}\label{teaser}
 }

\maketitle

\begin{abstract}


\end{abstract}

\keywordlist

\TOGlinkslist

\copyrightspace

\section{Introduction}
3D printing has the potential to dramatically reduce the required time and cost of
fabricating custom materials and objects. In particular, it allows the fabrication of materials with complex internal
structures that are difficult or impossible to manufacture with other technologies. Additive manufacturing could be
much more widely adopted and easier to use, given a high-fidelity translation from specifications of objects defined
in terms of their mechanical, optical, and appearance properties to output-device-specific computer
programs that produce the best possible approximation of these digital objects. However, this is still one of
the most important unsolved problems in digital direct manufacturing. Due to its combinatorial nature, this
problem becomes more and more challenging as the range of different printable materials increases. This paper
aims to provide an abstraction mechanism and a software framework for translating specifications to fabrication
instructions. It is possible to design a computationally efficient and general process for translating virtual objects
with desired mechanical, electrical, optical, and appearance properties into composite printable materials.
We will achieve this goal through a reduction-optimization-simulation approach.
\section{Previous Work}

\section{System Description}
The software framework (shown in Figure~\ref{fig:framework}) is composed of 
a set of software
modules that can be easily modified or exchanged. The first component
of this framework is an output-device-independent description API
that allows the user to specify any physical object in terms of its physical
properties such as geometry, mechanical behaviors, and appearance. 
The framework will also provide physically-based
simulation tools that accurately predict the behavior of our output device.
The goal of the optimization module is to search through
the space of possible device outputs and deliver one that best matches the
desired model. In order to evaluate the quality of the match between the
simulation output and the desired model we need to develop appropriate
comparison metrics. Finally, a central component of our framework is
a reduction model which makes the optimization problem tractable by
reducing the search space as much as possible.
The reduction model can specified by experts or deduced from a
set of examples. Next, we will describe each element of the framework in
more detail.
\begin{figure}
\caption{API overview}
\label{fig:framework}
\end{figure}
\subsection{Input API}
Our input API allows a designer to either
define material properties for sections of an object
or specify behavior of an object. 
The designer
can use functions to describe properties such as reflectance (e.g., BRDFs, BSSRDFs), 
stiffness, or other spatially varying properties. 
Although it is possible to define the object directly by only specifying material
properties, intuitive user
interfaces will aid in specifying the objects and their properties
in terms of intuitive metaphors.
For example, the elastic properties of an object can be specified by a set of
example displacements from the rest pose and the corresponding applied forces.
\subsection{Search Space Reduction and Optimization}
Describe four main components and some built-in libraries.
In our
framework, we propose an optimization component that
searches through the space of all possible device outputs
obtained by the simulator to nd the one that best reproduces
the desired object. Unfortunately, this search space
is usually high-dimensional. For example, when the printing
volume has N voxels and each of these voxels can be
assigned to one of M base materials, the search space
has NM dimensions. To overcome this problem, we reduce
the search space to a lower-dimensional space using
a reduction model. The goal of the reduction step is to
shrink the search space as much as possible but at the
same time not to prune out good solutions. The reduced
space allows for the use of standard optimization techniques such as simulated annealing, genetic algorithms,
Markov Chain Monte Carlo methods, and even an exhaustive search. The low-dimensional space produced by the
reducer can be defined in two different ways. First, a reduction model can be explicitly specified by an expert, for
example, by defining structures of base materials and rules how different base materials can be combined together.
The second way to specify a reduction model is implicit. In this case, the user provides a set of examples of
valid material structures. Then, a machine learning algorithm infers both basic material structures and the rules
for combining them. The optimization framework can be further improved by employing techniques that cluster
partial solutions that yield similar output properties [1], avoiding combinatorial explosion of the search space.
Furthermore, we can employ bounds on the solution based on physical constraints of the base materials.
\subsection{Simulation}
Another key step in the process of converting abstractions to physical output is being able to
simulate what an output device is going to generate given a well-defined input. A simulation allows the
user to preview how the output will look and behave. This simulation is a generalization of the print-preview function
available in word processing applications. As physical output generation might be costly or time-consuming, it
is extremely beneficial for users to be able to preview the output and make immediate changes to the design. That
means that the simulation must accurately predict the output. It's necessary to develop very efficient and accurate rendering
and finite element simulation packages. Since we run the simulation multiple times within the optimization method
we can cache and reuse partial simulation results, speeding up the evaluation process at least 10-100 times. Furthermore,
we will ensure that the simulation is accurate by measuring properties of the printing materials and using
these measurements to estimate the parameters of data-driven material models (e.g., for elasticity, reflectance).
\section{Experiments}
\subsection{Previous Work}
Subsurface scattering. Mechanical properties
\subsection{Textured Model}
Input: a textured 3D model and
	measured Albedo of print material.

Output: (Probably gray-scale) Material arranged for different printers in different formats: STL, fable,Gcode.

Printers: Our printer, Makerbot, Objet, maybe Zcorp.

\subsection{New Examples for Mechanical Properties}
Ball bounce to certain height. Loaded dice.

Printer:objet, ours.

\section{Possible Extension}
UI to specify input deformation.
\section{Conclusion}
\section*{Acknowledgements}

\bibliographystyle{acmsiggraph}
\bibliography{template}
\end{document}
