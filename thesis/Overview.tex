\chapter{Design Goals}
\label{chap:design}
\biglet{T}{he} design of my general translation framework is guided by the following principles:

\begin{itemize}

\item \textbf{Modularity:} Spec2Fab translators are complicated both algorithmically and from a software engineering point of view. To combat this, any proposed framework must break the problem into a manageable number of small, reusable building blocks. 

\item \textbf{Extensibility:} Developers must be able to add their own building blocks to the system. This allows the system to grow in conjunction with the capabilities of newer 3D printers.
\vspace{-0.25\baselineskip}

\item \textbf{Device Independence:} Spec2Fab translators should be device independent. They should be easily adaptable to different types of 3D printers.
\item \textbf{Input Geometry Independence:} Spec2Fab translators should be geometry independent. For example, a process for applying a texture to a 3D printed object should work for \textbf{any} object.
\vspace{-0.25\baselineskip}
\end{itemize}

I aim to separate the Spec2Fab process into two phases, the process configuration phase and the process use phase. The process configuration phase is typically done once by a skilled developer who  constructs a Spec2Fab translator. The process use phase is typically performed multiple times by an end-user who is only required to provide an object specification (e.g., object geometry and deformation properties) and a target device. 

	The process configuration phase produces a Spec2Fab translator which will assign a desired volumetric material distribution to a user supplied input geometry given a user specified goal. A developer can describe this phase using two new data structures, the \emph{reducer tree} and \emph{tuner network}. The \emph{reducer tree} parameterizes a volumetric material assignment using a small set of \emph{geometry} and \emph{material nodes} while the \emph{tuner network} is used to describe an optimization process as a connected set of \emph{tuner} objects.  Both \emph{nodes} and \emph{tuners} can be easily recombined and reused thus making my framework highly modular. Furthermore,  \emph{nodes} and \emph{tuners} define abstract interfaces and thus developers can easily add new types of each. This makes my framework extensible.
	
	\emph{Reducers} and \emph{tuners} are chosen to be independent of printer capabilities. 
	Instead, a user can account for printer type by altering the available materials that the translator may assign to the input geometry.  This is important since it grants the \emph{reducer-tuner} model  device independence. Finally, the \emph{geometry nodes} of the \emph{reducer tree} are designed to function irrespective of input shape. Since the Spec2Fab translators represented by my data structures use a composition of these node types, they are geometry independent. 
	
	In the following sections I will describe the \emph{reducer tree} and the \emph{tuner network}, their constituent components and the mechanisms by which they interact. 
