%%
%% MIT Thesis - Chapter 1
%%

\renewcommand{\figdir}{figures/chap1}
%
\chapter{Introduction}
\label{chap:intro}
%
\biglet{3}{D} printing receives a lot of attention as it aims to democratize fabrication.
The ever expanding range of printing materials allows for fabrication of complex objects with spatially varying appearance, optical characteristics, and mechanical properties.
One of the most important unsolved problems in this area is how to compute an object's material composition from a functional or behavioral description. I will call this process \em specification to fabrication translation (Spec2Fab). 
The goal of this work is to provide a convenient abstraction for specifying such translators. This is necessary to move past the current direct specification model of 3D printing.

Today,  3D printing of an object requires a material be directly specified for each voxel inside the object volume. This approach is fraught with difficulties. First, 3D printable models become specific to a single printer type, i.e., the models are built from materials provided by a given printer. Consider the inconvenience that would result from  word processing documents being compatible with specific 2D printers. Second, working directly with printing materials rather than material properties is extremely challenging for users. Imagine the difficulty in finding the right combination of printing materials that would provide a specific color, stiffness, or refractive index.

My work is motivated by the recent research efforts in the computer graphics community to create specific instances of the translation process, for example, subsurface scattering~\cite{Hasan:2010:PRO,Dong:2010:FSS} or deformation properties~\cite{Bickel:2010:DAF}. However, each of these instances is a custom, monolithic solution which is difficult to extend, combine, or modify. Our main insight is that all these process instances share a similar structure. First, they rely on the ability to accurately simulate the physical properties of an object given its geometry and material assignment. They use this simulation within an optimization framework to search the space of all possible material assignments in order to find the one that best reproduces the desired properties. Due to the combinatorial nature of the search space the naive optimization approach is not tractable. For example, when the printing volume has $N$ voxels and each of these voxels can be assigned to one of $M$ base materials, the search space has $N^M$ dimensions. To overcome this problem, the search space is reduced to a lower-dimensional space using a reduction model. The goal of the reduction step is to aggressively shrink the search space in a domain-specific manner such that it still contains good approximations to the optimal solution. This search space reduction combined with the right choice of the optimization algorithm delivers a computationally tractable approximation.

The reduction-optimization structure suggests that it is possible to provide a more general abstraction mechanism for translating 3D models to printer and material-specific representations. In this paper we take the first step in achieving this goal. Our solution relies on two novel data structures which are designed to aid the fabrication process. The \emph{reducer tree} is a tree-based data structure that allows us to parameterize the space of material assignments. The \emph{tuner network} is a data structure for specifying the optimization process. Our solution also provides an API for specifying the desired object, setting up the simulation, and defining parameters for the \emph{reducer tree} and \emph{tuner network}. In general, our framework simplifies the construction of new computational fabrication algorithms. More specifically, different components of the process can be easily replaced and other components easily reused. Various optimization strategies can also be explored with lower implementation burden. In order to show these advantages, I will illustrate how existing computational design processes fit into this framework and how they can be combined. I will demonstrate the results of these algorithms on a variety of different examples fabricated using 3D printers.
